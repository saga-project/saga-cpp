
\subsection{Overview}

 The SAGA C++ sources include an adaptor generator, which allows to
 easily create stubs for custom adaptors.  The script is located in

 \shift |adaptors/generator/generator.pl|

 and is installed into |\$SAGA_LOCATION/bin/|. Calling that script
 without any arguments will print a help screen, which provides a
 number of details on the command line arguments etc.  
 
 The exemplary
 shell session shown below demonstrates the use of the adaptor
 generator, and results in a complete file adaptor:

 \begin{mycode}
   # cd saga-core-src/adaptors/

   # ./generator/saga-adaptor-generator.pl -s ssh -n sshfs -t file -d .

     suite:             ssh
     type:              file
     ftype:             sshfs_file
     name:              sshfs
     directory:         ./ssh/ssh_sshfs_file

     copying files:     ...
     fixing file names: ..............
     fixing files:      ....................

     You can now cd to ./ssh/ssh_sshfs_file, 
     and run 'make; make install'.  

     Note that you need to set SAGA_LOCATION before,
     and point it to your SAGA installation tree.

   # cd ssh/ssh_sshfs_file

   # make
     compiling    ssh_sshfs_file_adaptor.o
     compiling    ssh_sshfs_file_dir_impl.o
     compiling    ssh_sshfs_file_dir_nsdir_impl.o
     compiling    ssh_sshfs_file_dir_nsentry_impl.o
     compiling    ssh_sshfs_file_dir_perm_impl.o
     compiling    ssh_sshfs_file_file_impl.o
     compiling    ssh_sshfs_file_file_nsentry_impl.o
     compiling    ssh_sshfs_file_file_perm_impl.o
     liblinking   libsaga_adaptor_ssh_sshfs_file.so
     liblinking   libsaga_adaptor_ssh_sshfs_file.a (static)

   # make install
     installing   lib
     installing   lib (static)
     installing   adaptor ini

 \end{mycode}

  When running any SAGA application, that adaptor will get loaded and
  will receive requests to perform remote operations.  That can be
  confirmed by setting |SAGA_VERBOSE| in the application environment
  (see SAGA installation guide).

  Of course, that adaptor will not be able to perform any meaningful
  operation -- it is just a stub, and will simply throw
  |NotImplemented| exceptions for all calls.  However, it is now
  straight forward to fill the stub with the respective functionality.  

  \F{Need to add details about class hierarchy, adaptor data, and
  instance data. Also, some details about adaptor makefiles, configure
  support etc might be useful.}

  Note that each adaptor will end up in a separate shared library.
  Since a typical SAGA installation will use multiple adaptors, and
  thus multiple adaptor libraries will be loaded into the same address
  space, the adaptor programmer needs to make sure to choose unique
  symbole names (i.e. to choose a unique and descriptive symbol name
  space), to avoid runtime symbol clashes.

