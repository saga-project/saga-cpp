

\subsection{Quick Introduction}

 SAGA error handling is exception-based: a rather flat hierarchy of 14
 exception types inherit the properties of the general
 |saga::exception| class.  That allows to inspect the exception for
 its exact type, and for the detailed error message(s) associated with
 this exception.

 \HINT{As our SAGA implementation is adaptor-based, a single API call
 could actually return more than one exception at the same time.  That
 is rendered by returning the most specific exception, whose error
 message then also contains the error messages from all the other
 exceptions, usually one per adaptor.}


\subsection{Reference}

 \begin{mycode}[label=Prototype: saga::exception class and derivates]

  namespace saga 
  {
  class exception : public std::exception
  {
  public:
     ~exception() throw() {}

     // Gets the message associated with the exception 
     char const* get_message() const throw() 

     // an alias for get_message
     char const* what() const throw() 

     // get type of exception
     saga::error get_error () const  

     // Gets the SAGA object associated with exception. 
     saga::object get_object () const throw() 
  }; 



  // parameter related exceptions
  class parameter_exception   : public saga::exception
  class incorrect_url         : public saga::parameter_exception
  class bad_parameter         : public saga::parameter_exception

  // state related exceptions
  class state_exception       : public saga::exception
  class already_exists        : public saga::state_exception
  class does_not_exist        : public saga::state_exception
  class incorrect_state       : public saga::state_exception
  class timeout               : public saga::state_exception

  // security related exceptions
  class security_exception    : public saga::exception
  class permission_denied     : public saga::security_exception
  class authorization_failed  : public saga::security_exception
  class authentication_failed : public saga::security_exception

  // general exceptions
  class no_success            : public saga::exception
  class not_implemented       : public saga::exception
  
 \end{mycode}


\subsection{Details}

 Error handling in SAGA is completely exception-based\footnote{The
 SAGA specification allows for an additional \T{error\_handler}
 interface to be implemented by SAGA objects---that seemed
 unnecessary for object-oriented languages such as C++.}.  The
 following exceptions exist (note that the exception class names and
 the |saga::error| enums, returned by |exception.get_error()| have
 identical names):

 \begin{description}

  \item{\B{NotImplemented}}
  
    A method is specified in the SAGA API, but is not provided
    by this specific SAGA implementation\footnote{In particular,
    the method is not provided by any of the available adaptors,
    see Section~\ref{sec:design}.}.
    

  \item{\B{IncorrectURL}}
  
   This exception indicates that a URL argument could not be
   handled.  For example, this implementation may be unable to
   handle the specified protocol, or the access to the specified
   entity, via the given protocol, is impossible.


  \item{\B{BadParameter}}

   This exception indicates that at least one of the parameters
   of the method call is ill-formed, invalid, out of bounds or,
   otherwise, not usable.  The error message gives specific
   information on what parameter caused the exception, and why.


  \item{\B{AlreadyExists}}

   This exception indicates that an operation cannot succeed
   because an entity to be created already exists, and cannot be
   overwritten.  Explicit flags on the method invocation may
   allow the operation to succeed, e.g., if they indicate that
   Overwrite is allowed.


  \item{\B{DoesNotExist}}

   This exception indicates that an operation cannot succeed
   because a required entity is missing.  Explicit flags on the
   method invocation may allow the operation to succeed, e.g.,
   if they indicate that Create is allowed.


  \item{\B{IncorrectState}}

   This exception indicates that the object a method was called
   upon is in a state where that method cannot possibly succeed.
   A change of state might allow the method to succeed with the
   same set of parameters.


  \item{\B{PermissionDenied}}

   An operation failed because the identity used for the
   operation did not have sufficient permissions to successfully
   perform the operation.
   

  \item{\B{AuthorizationFailed}}

   An operation failed because none of the available contexts of
   the used session could be used to access the given resource.
   In contrast to the \T{PermissionDenied}, this exception
   usually indicates an error on the administrative level, which
   usually cannot be fixed by the end user.


  \item{\B{AuthenticationFailed}}
  
   An operation failed because none of the available session
   contexts could successfully be used for authentication.  That
   exception usually indicates invalid or outdated security
   credentials.


  \item{\B{Timeout}}

   This exception indicates that a remote operation was not
   completed successfully, because the network communication or
   the remote service timed out.  This exception is never thrown
   if a timed \T{wait()} or similar method times out, as that is
   not an error condition.


  \item{\B{NoSuccess}}

   This exception indicates that an operation failed for any
   other reason.  The exception message may, or may not, contain
   some more details about the cause of the error.

 \end{description}

 Even for simple SAGA operations, the implied Grid interactions
 can be fairly complex and may invoke multiple remote
 operations.  It may thus happen that multiple exceptions apply
 during the execution of the method.  In such cases, the \B{most
 specific exception} is thrown (the list of exceptions above is
 ordered, with the most specific exception up front).  
 
 Also, as the above Reference section shows, the exceptions are
 ordered in an exception hierarchy.  The application can thus
 try to catch whole classes of exceptions, e.g., all security
 related exceptions, or all state related exceptions.  The code
 example below shows, however, how the lower level exceptions
 can be accessed for debugging purposes:

 \begin{mycode}
  try
  {
    saga::url u ("/path/which/does/not/exist");
    saga::filesystem::file f (u);
  }
  catch ( saga::state_exception const & e )
  {
    switch ( e.get_error () )
    {
      // handle does not exist
      case saga::DoesNotExist:
      {
        std::cout << "file does not exist"
                  << std::endl;
        exit (-1);
        break;
      }

      // generic handler for state related problems
      default:
      {
        std::cout << "some other saga state exception caught: "
                  << std::string (e.what ())
                  << std::endl;
        exit (0);
        break;
      }
    }
  }
\end{mycode}

