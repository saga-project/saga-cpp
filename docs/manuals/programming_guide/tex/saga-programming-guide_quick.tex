
 So, you are one of the impatient readers who dread long dry user
 manuals?  No problem: read this chapter, and you will have a good
 starting point to use SAGA.  After all, the \B{S} in \B{S}AGA stands
 for \B{Simple}!  The remainder of the document provides considerable
 more details about the API, but you can defer the later chapters until
 you have indeed the need for more information.

 % \subsection{Structure of the API}

  It is important to understand that the API consists of two parts:
  the \LF, and the API packages.  The packages are what you are most
  likely interested in, because they provide the means to interact with the
  Grid: they start jobs, copy files, perform remote procedure calls, etc.
  The \LF provides the syntactic and semantic expressiveness to
  control \I{how} these actions are expressed (syntax) and performed
  (semantics).  For example, the session management in the \LF part
  tells you how to specify the security constraints of your actions,
  while the task model from the \LF determines how you can express
  syncronous versus asynchronous actions.

  That all sounds rather theoretical: let us dive into some examples!
  The first one allows you to copy a file\footnote{For the sake of brevity, we leave error and sanity checks out of the examples}:

  \begin{mycode}[label=File copy]
  #include <saga/saga.hpp>

  int main (int argc, char** argv)
  {
    // do a file copy
    saga::url u(argv[1]);
    saga::filesystem::file f (u);
    f.copy (saga::url (argv[2]));
  }
  \end{mycode}

  Isn't that simple?  Three lines, and your file is copied!
  |file.copy()| is provided by the |file| class in the
  |saga::filesystem| package.  That is a \I{functional} SAGA call.
  So, how will \I{non-functional} properties of that call enter the
  picture?  Let's try to have the same call asynchronously:

  \begin{mycode}[label=File copy (async version)]
  #include <saga/saga.hpp>

  int main (int argc, char** argv)
  {
    // run a file copy asynchronously
    saga::url u(argv[1]);
    saga::filesystem::file f (u);
    saga::task t = f.copy <saga::task::Async> (saga::url (argv[2]));

    // do something else

    // wait for the copy task to finish
    t.wait ();
  }
  \end{mycode}

  That is almost the same.  In particular, the signature for the
  |file.copy()| method is the same.  It is, however, now qualified as
  |saga::task::Async|, and returns a |saga::task| instance.  That
  (stateful) task represents the asynchronous call.  You can think of
  the synchronous qualification as the default, which can be left out.

  Another \LF package deals with monitoring, and allows, for example,
  to get notifications if the task finishes.  Yet another \LF package
  allows to specify security credentials for the copy operation, and
  so on and so forth.

  So, that is the general picture: SAGA packages provide you with means
  to interact with the Grid environment, and the \LF packages
  determine how these actions are executed, syntactically and
  semantically, altering the funtional method's performance.


  The remainder of the document will give more details, first about
  the \LF packages of SAGA, and then about the functional packages. 
  Each section starts with a quick introduction (which should be enough to
  get you going in most cases), followed by a reference section, and
  by a more detailed discussion where appropriate.  Finally, a number
  of advanced topics are discussed, which are probably not of your interest
  initially, but they are certainly useful concepts as soon as your application
  reaches a certain complexity itself.

  Before diving into the API itself, we will shortly describe how to
  compile and run SAGA applications, so that the reader will be able
  to actively follow the coding examples and excercises.

