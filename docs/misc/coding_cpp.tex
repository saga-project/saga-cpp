
\documentclass{article}

\usepackage{ifpdf}
\usepackage{subfigure}

\ifpdf
  \usepackage[pdftex]{graphicx}
  \usepackage[pdftex]{hyperref}
  \graphicspath{{Pics/}}
  \DeclareGraphicsExtensions{.pdf, .png, .jpg}
\else
  \usepackage{graphicx}
  \usepackage[hypertex]{hyperref}
  \graphicspath{{Pics/}}
  \DeclareGraphicsExtensions{.ps, .eps}
\fi

\usepackage{srcltx}
\usepackage{wrapfig}

\newcommand{\I}{\textit}
\newcommand{\B}{\textbf}
\newcommand{\T}{\texttt}

\newcommand{\F}[1]{\B{FIXME: #1}}

\begin{document}

\title{C++ Coding Guidlines for the SAGA C++ Reference Implementation
       'SAGA-A'}
\author{Andre Merzky and Hartmut Kaiser\footnotemark}
\maketitle

\footnotetext{based on the GridLab style Guide by Tom Goodale}

\begin{abstract}

  This Documents presents the C++ coding conventions as used by the
  SAGA C++ Reference Implementation.  They should be followed by
  \textbf{all} contributing parties.

\end{abstract}

\tableofcontents

\section{C++}

These coding conventions are from the GridLab Coding conventions. See
\url{http://www.gridlab.org/WorkPackages/techboard/Docs/coding_cpp.pdf/}.

\subsection{File Organization}

A file consists of sections that should be separated by blank lines
and an optional comment identifying each section.

\subsection{File Names}

File names should be all lowercase, with the extension \texttt{.hpp}
for include files, and \texttt{.cpp} for source files. The file
names should reflect the functionality defined/implemented in that
file.  Files with logical connection (e.g. pairs of header and source
files) should reflect that connection in their names wherever
possible.  Files belonging to the same moduleshould reflect that
dependency by a short uniq prefix to the filename, followed by an
underscore.  Following examples illustrate this convention:

\begin{quote}
\begin{verbatim}
 Makefile                  io_base.cpp
 server.cpp                io_base.hpp
 server_ioplug.hpp         io_file.hpp
 server_ioplug.c           io_file.cpp
 io_cwrapper.hpp           io_stream.hpp
 io_cwrapper.c             io_stream.cpp
\end{verbatim}
\end{quote}


\subsection{Source Files}

C++ source files have the following ordering: 

\begin{itemize}
  \item Beginning comments (see ``Beginning Comments'' on
        \pageref{sec:cguide:beginning_comments}) 
  \item \#include statements
  \item \#defines.
  \item local data type definitions.
  \item local (static) function prototypes.
  \item local (static) data.
  \item externally visible functions.
  \item local (static) functions.
\end{itemize}

\paragraph{Beginning Comments}
\label{sec:cguide:beginning_comments}

All source files should begin with a comment that describes the file
and its contents, and a copyright statement.  The \T{LICENSE} line in the
example below is a placeholder for an automatically inserted license
text.  That insertion is performed while packaging the code for
distribution.

\begin{quote}
\begin{verbatim}
/** 
 *
 * Brief description of contents of file.
 * 
 * Long description
 * 
 * Copyright notice.
 */

 /*** LICENSE ***/

\end{verbatim}
\end{quote}

\textbf{Note:}
This is a documentation comment -- see section
\ref{sec:cguide:doc_comments} for details.

\subsection{Header Files}

All header files should begin with a comment that describes the file
and its contents, and a copyright statement.

\begin{quote}
\begin{verbatim}
/** 
 * Brief description of contents of file.
 * 
 * Long description
 * 
 * Copyright notice.
 */

 /*** LICENSE ***/

\end{verbatim}
\end{quote}

\textbf{Note:}
This is a documentation comment -- see section
\ref{sec:cguide:doc_comments} for details.

To protect against multiple inclusion of headers, the contents of a
header file should be protected by a \#ifndef ... \#endif pair.

\begin{quote}
\begin{verbatim}
#ifndef NAME_OF_HEADER_FILE_IN_CAPITALS_HPP
#define NAME_OF_HEADER_FILE_IN_CAPITALS_HPP

...body of header file...

#endif // NAME_OF_HEADER_FILE_IN_CAPITALS_HPP

\end{verbatim}
\end{quote}


\subsection{Indentation}

Two spaces should be used as the unit of indentation.  Tabs should not
be used as inconsistent use of tabs and spaces leads to difficulties
when using ``diff'' or other tools to compare files.

\subsubsection{Line Length}

Avoid lines longer than 80 characters, since they're not handled well
by many terminals and tools.
 

\subsubsection{Wrapping Lines}

When an expression will not fit on a single line, break it according
to these general principles:

\begin{itemize}
\item Break after a comma.  
\item Break before an operator.  
\item Prefer higher-level breaks to lower-level breaks.  
\item Align the new line with the beginning of the expression at the same
      level on the previous line.
\item If the above rules lead to confusing code or to code that's squished
      up against the right margin, just indent 8 spaces instead.
\end{itemize}

Here are some examples of breaking function calls:

\begin{quote}
\begin{verbatim}
Function (longExpression1, longExpression2, longExpression3, 
          longExpression4, longExpression5);
 
var = Function (longExpression1,
                Function2 (longExpression2,
                           longExpression3)); 
\end{verbatim}
\end{quote}

Following are two examples of breaking an arithmetic expression. The
first is preferred, since the break occurs outside the parenthesised
expression, which is at a higher level.

\begin{quote}
\begin{verbatim}
LongName1 = LongName2 * (LongName3 + LongName4 - LongName5)
            + 4 * LongName6; /* PREFER */

LongName1 = LongName2 * (LongName3 + LongName4
                         - LongName5) + 4 * LongName6; /* AVOID */
\end{verbatim}
\end{quote}

Following are two examples of indenting function declarations. The
first is the conventional case. The second would shift the second and
third lines to the far right if it used conventional indentation, so
instead it indents only 8 spaces.  Recommended is the third variant.

\begin{quote}
\begin{verbatim}
/* CONVENTIONAL INDENTATION */
Function (int AnArg, double AnotherArg, char *YetAnotherArg,
          int *AndStillAnother) 
{
    ...
}

/*INDENT 8 SPACES TO AVOID VERY DEEP INDENTS */
static ReallyLongFunctionName (int AnArg,
        double anotherArg, char *YetAnotherArg,
        int *AndStillAnother) 
{
    ...
}

/* PREFERRED: PUT EACH ARG ON OWN LINE   */
static ReallyLongFunctionName (int      AnArg,
                               double   anotherArg, 
                               char   * YetAnotherArg,
                               int    * AndStillAnother) 
{
    ...
}
\end{verbatim}
\end{quote}

Here are three acceptable ways to format ternary expressions:

\begin{quote}
\begin{verbatim}
alpha = (aLongBooleanExpression) ? beta : gamma;  

alpha = (aLongBooleanExpression) ? beta
                                 : gamma;  

alpha = (aLongBooleanExpression)
      ? beta 
      : gamma;  
\end{verbatim}
\end{quote}


\subsection{Comments}
\label{sec:cguide:comments}

Programs can have two kinds of comments: implementation comments
and documentation comments. Implementation comments are those 
delimited by /*...*/, or single lines comments starting with //. 
Documentation comments are delimited by
/**...*/, and can be extracted to HTML files using the
doxygen tool.

Implementation comments are mean for commenting out code or for
comments about the particular implementation. Doc comments are meant
to describe the specification of the code, from an implementation-free
perspective, to be read by developers who might not necessarily have
the source code at hand.

Comments should be used to give overviews of code and provide
additional information that is not readily available in the code
itself. Comments should contain only information that is relevant to
reading and understanding the program. For example, information about
how a corresponding package is built or in what directory it resides
should not be included as a comment.

Discussion of nontrivial or nonobvious design decisions is
appropriate, but avoid duplicating information that is present in (and
clear from) the code. It is too easy for redundant comments to get out
of date. In general, avoid any comments that are likely to get out of
date as the code evolves.

\textbf{Note:}

The frequency of comments sometimes reflects poor quality of
code. When you feel compelled to add a comment, consider rewriting the
code to make it clearer.

Comments should not be enclosed in large boxes drawn with asterisks or
other characters. Comments should never include special characters
such as form-feed and backspace.

\subsubsection{Implementation Comment Formats}

Programs can have four styles of implementation comments: block,
single-line, trailing, and end-of-line. 

\paragraph{Block Comments}
\label{sec:cguide:block_comments}

Block comments are used to provide descriptions of files, methods,
data structures and algorithms. Block comments may be used at the
beginning of each file and before each method. They can also be used
in other places, such as within functions. Block comments inside a
function should be indented to the same level as the code
they describe.

A block comment should be preceded by a blank line to set it apart
from the rest of the code.

\begin{quote}
\begin{verbatim}
/*
 * Here is a block comment (prefferred).
 * Here is a block comment (prefferred).
 * Here is a block comment (prefferred).
 */

// Here is anmother block comment.
// Here is anmother block comment.
// Here is anmother block comment.

\end{verbatim}
\end{quote}

See also ``Documentation Comments'' on page \pageref{sec:cguide:doc_comments}. 

\paragraph{Single-Line Comments}
\label{sec:cguide:sline_comments}

Short comments can appear on a single line indented to the level of
the code that follows. If a comment can't be written in a single line,
it should follow the block comment format (see section
\ref{sec:cguide:block_comments}). A single-line comment should be preceded by
a blank line. Here's an example of a single-line comment in C code
(also see ``Documentation Comments'' on page
\pageref{sec:cguide:doc_comments}):

\begin{quote}
\begin{verbatim}
if (condition) 
{
  // Handle the condition. 
  ...
}
\end{verbatim}
\end{quote}

\paragraph{Trailing Comments}
\label{sec:cguide:trailing_comments}

Very short comments can appear on the same line as the code they
describe, but should be shifted far enough to separate them from the
statements. If more than one short comment appears in a chunk of code,
they should all be indented to the same tab setting.

Here's an example of a trailing comment in C code: 

\begin{quote}
\begin{verbatim}
  if ( a == 2 ) 
  {
    a = TRUE;          // special case
  } 
  else 
  {
    a = isPrime (a);   // works only for odd a
  }
\end{verbatim}

\end{quote}

\subsubsection{Documentation Comments }
\label{sec:cguide:doc_comments} 

For further details, see ``The Doxygen Manual''
which includes information on the doc comment tags (@return, @param,
@see):

\begin{quote}
\url{http://www.doxygen.org}
\end{quote}

Doxygen comments describe C functions, structures, enums, unions,
etc. Each doc comment is set inside the comment delimiters /**...*/,
with one comment per class, interface, or member. This comment should
appear just before the declaration:

\begin{quote}
\begin{verbatim}
/**
 * The Example function provides ...
 */
void Example (void)
{ ...

\end{verbatim}
\end{quote}

The first line of doc comment (/**) is not indented relative to the
surrounding block; subsequent doc comment lines each have 1
space of indentation (to vertically align the asterisks).

If you need to give information that isn't appropriate for
documentation, use an implementation block comment (see section
\ref{sec:cguide:block_comments}) or single-line (see section
\ref{sec:cguide:sline_comments}) comment immediately
\emph{before} the declaration. For example, details about the
implementation of a function should go in in such an implementation
block comment \emph{following} the doc comment for the function, not
in the function doc comment.


\subsection{Declarations}

\subsubsection{Number Per Line}

One declaration per line is recommended since it encourages commenting. In other words, 

\begin{quote}
\begin{verbatim}
int level;   // indentation level
int size;    // size of table 
\end{verbatim}
\end{quote}

is preferred over 

\begin{quote}
\begin{verbatim}
int level, size;
\end{verbatim}
\end{quote}

Do not put different types on the same line. Example: 

\begin{quote}
\begin{verbatim}
int foo,  fooarray[SIZE]; // WRONG!
\end{verbatim}
\end{quote}
 
\textbf{Note:}

The examples above use one space between the type and the identifier.
Groups of declarations and comments should be lined up where possible
(with spaces, not tabs), e.g.:

\begin{quote}
\begin{verbatim}

int     level;	      // indentation level              
int     size;	        // size of table                  
Object  currentEntry;	// currently selected table entry 

\end{verbatim}
\end{quote}
 

\subsubsection{Initialisation}

Variables should not be left un-initialized, or only as short as
possible.


\subsubsection{Placement}

Put declarations only at the beginning of blocks. (A block is any code
surrounded by curly braces ``\{`` and ``\}''.) Don't wait to declare
variables until their first use; it can confuse the unwary programmer
and hamper code portability within the scope.

\begin{quote}
\begin{verbatim}
void MyFunction(...) 
{
  int int1 = 0;   // beginning of block

  if ( condition ) 
  {
    int int2 = 0; // beginning of "if" block
    ...
  }
}

\end{verbatim}
\end{quote}

The one exception to the rule is indices of for loops in C++, which 
can be declared in the for statement:

\begin{quote}
\begin{verbatim}

for (int i = 0; i < MaxLoops; i++) 
{ 
  ... 
}

\end{verbatim}
\end{quote}

Avoid local declarations that hide declarations at higher levels. For
example, do not declare the same variable name in an inner block:

\begin{quote}
\begin{verbatim}
int MyFunction (...) 
{
  int count;
  
  if ( condition ) 
  {
    int count; // AVOID!
    ...
  }

  ...
}
\end{verbatim}
\end{quote}

\subsubsection{Public versus Private Declarations}

When defining inheritance, class members and class methods, the
minimal possible set should be public, the minimal possible set should
be protected, and the maximal possible set should be private.  Always
use the keywords public, protected and private explicitely, and don't
rely on the defaults.


\subsubsection{Function Declarations}

When coding functions the following formatting rules should be
followed:

\begin{itemize}
\item All functions should be preceded by a documentation comment
      describing the function, its arguments and return code(s).
\item Open brace ``\{`` appears at the beginning of the line following
      the declaration statement  
\item Closing brace ``\}'' starts a line by itself.
\end{itemize}

\begin{quote}
\begin{verbatim}

/** The sample function.
 *  This function calculates the sample thingy.
 *  Some more description.
 *
 * @param i The first argument.
 * @param j the second argument.
 * 
 * @return The sample value.
 */
int sample (int i, int j) 
{
  return (i + j);
}

\end{verbatim}
\end{quote}

\subsection{Statements}
\subsubsection{Simple Statements}

Each line should contain at most one statement. Example: 

\begin{quote}
\begin{verbatim}

argv++;           // Correct 
argc--;           // Correct 
argv++; argc--;   // AVOID! 

\end{verbatim}
\end{quote}


\subsubsection{Compound Statements}

Compound statements are statements that contain lists of statements
enclosed in braces ``\{ statements \}``. See the following sections
for examples.


\begin{itemize}
\item The enclosed statements should be indented one more level than
      the compound statement.  

\item The opening brace should be on a line by itself following the
      line that begins the compound statement, indented to the same
      level as that line; the closing brace should begin a line and be
      indented to the beginning of the compound statement.

\item Braces are used around all statements, even single statements,
      when they are part of a control structure, such as a if-else or
      for statement. This makes it easier to add statements without
      accidentally introducing bugs due to forgetting to add braces.

\end{itemize}
 
\begin{quote}
\begin{verbatim}

void sample (int t)
{
  for ( int i = 0; i < t; i++ )
  {
    if ( 3 == t )
    {
      return;
    }
  }

  return;
}

\end{verbatim}
\end{quote}

\subsubsection{return Statements}

Unless a return value is a single word expression, the return
statement should use parentheses to make the return value more
obvious. Example:

\begin{quote}
\begin{verbatim}
return;

return size;

return (fix_size (size));

return (size ? size : defaultSize);

\end{verbatim}
\end{quote}


\subsubsection{if, if-else, if else-if else Statements}

The if-else class of statements should have the following form: 

\begin{quote}
\begin{verbatim}
if ( condition ) 
{
  statements;
}

if ( condition ) 
{
  statements;
} 
else 
{
  statements;
}

if ( condition ) 
{
  statements;
} 
else if ( condition ) 
{
  statements;
} 
else
{
  statements;
}
\end{verbatim}
\end{quote}

\textbf{Note:}

if statements always use braces \{\}. Avoid the following error-prone
form: 

\begin{quote}
\begin{verbatim}
if ( condition ) // AVOID!
    statement;
\end{verbatim}
\end{quote}
 
\subsubsection{for Statements}

A for statement should have the following form: 

\begin{quote}
\begin{verbatim}
for ( initialization; condition; update ) 
{
  statements;
}
\end{verbatim}
\end{quote}

An empty for statement (one in which all the work is done in the
initialization, condition, and update clauses) should have the
following form, which makes its purpose more obvious:

\begin{quote}
\begin{verbatim}
for ( initialization; condition; update ) ;
\end{verbatim}
\end{quote}

When using the comma operator in the initialization or update clause
of a for statement, avoid the complexity of using more than three
variables. If needed, use separate statements before the for loop (for
the initialization clause) or at the end of the loop (for the update
clause).

\subsubsection{while Statements}

A while statement should have the following form: 

\begin{quote}
\begin{verbatim}
while ( condition ) 
{
  statements;
}

\end{verbatim}
\end{quote}

An empty while statement should have the following form: 

\begin{quote}
\begin{verbatim}
while ( condition ) ;
\end{verbatim}
\end{quote}

\subsubsection{do-while Statements}

A do-while statement should have the following form: 

\begin{quote}
\begin{verbatim}
do 
{
  statements;
} while ( condition );
\end{verbatim}
\end{quote}

\subsubsection{switch Statements}

A switch statement should have the following form: 

\begin{quote}
\begin{verbatim}
switch ( condition ) 
{
  case ABC:
    statements;
    // falls through

  case DEF:
    statements;
    break;

  case XYZ:
    statements;
    break;

  default:
    statements;
    break;
}
\end{verbatim}
\end{quote}

Every time a case falls through (doesn't include a break statement),
add a comment where the break statement would normally be. This is
shown in the preceding code example with the /* falls through */
comment.

Every switch statement should include a default case. The break in the
default case is redundant, but it prevents a fall-through error if
later the default is changed to a specific case and a new default is
introduced.

For longer code blocks, brackets can be used for the indivudual cases,
as in the following example, but should be used always or not at all
in a single case block:

\begin{quote}
\begin{verbatim}
switch ( condition ) 
{
  case ABC:
  {
    statements;
    statements;
    statements;
    // falls through
  }

  case DEF:
  {
    statements;
    break;
  }

  case XYZ:
  {
    statements;
    break;
  }

  default:
  {
    statements;
    break;
  }
}
\end{verbatim}
\end{quote}
\subsection{White Space}

\subsubsection{Blank Lines}

Blank lines improve readability by setting off sections of code that
are logically related. 

Two blank lines should always be used in the following circumstances: 

\begin{itemize}
\item Between sections and functions of a source file  
\end{itemize}

One blank line should always be used in the following circumstances: 

\begin{itemize}
\item Between the local variable declarations in a block and its first statement  
\item Before a block (see section \ref{sec:cguide:block_comments}) or single-line
      (see section \ref{sec:cguide:sline_comments}) comment
\item Between logical sections inside a function to improve readability 
\end{itemize}
 
\subsubsection{Blank Spaces}

Blank spaces should be used in the following circumstances: 

\begin{itemize}
\item A blank space should appear after commas in argument lists.  
\item All binary operators should be separated from their
      operands by spaces. Blank spaces should never separate unary
      operators such as unary minus, increment (``\T{++}''), and decrement
      (``\T{--}'') from their operands. Example: \T{a + b; i++;}.
\item brackets of if, for, while, and do operators should have leading
      and trailing blanks (``\T{if ( 3 == i )}'', not ``\T{if (3 == i)}''). 

\end{itemize}
\begin{quote}
\begin{verbatim}

    a +=  c + d;
    a  = (a + b) / (c * d);
    
    while ( d++ = s++ ) 
    {
      n++;
    }
\end{verbatim}
\end{quote}


\begin{itemize}
\item The expressions in a for statement should be separated by blank spaces. Example: 
\end{itemize}

\begin{quote}
\begin{verbatim}

    for ( expr1; expr2; expr3 )

\end{verbatim}
\end{quote}

\begin{itemize}
\item Casts should be followed by a blank space. Examples: 
\end{itemize}

\begin{quote}
\begin{verbatim}
    MyFunction1 ((char) aNum, (double) x);
    MyFunction2 ((int) (cp + 5), ((int) (i + 3) + 1);
\end{verbatim}
\end{quote}

\begin{itemize}
\item Template brackets should be separated by a blank space. Examples: 
\end{itemize}

\begin{quote}
\begin{verbatim}
    template <class T> MyFunction (void);
    std::vector <std::pair <std::string, std::string> > map;
\end{verbatim}
\end{quote}

\begin{itemize}
\item Function invokations should have a blank before their brackets.
Example:
\end{itemize}

\begin{quote}
\begin{verbatim}
    a = get_a(b, c);  // AVOID
    a = get_a (b, c); // PREFER
\end{verbatim}
\end{quote}

\subsection{Naming Conventions}

Naming conventions make programs more understandable by making them
easier to read.

\begin{tabular}{|l|p{42ex}|}
\hline
  Externally visible functions and classes
& 
  if no namespaces are used, they should have a short prefix uniquely
  identifying the module, followed by an underscore, followed by the
  rest of the identifier which should consist of words, separated by
  underscores, e.g. \emph{saga\_find\_resource}.
\tabularnewline
\hline
  Static functions and classes
&
  should follow the same convention as externally visible functions,
  but need not to have the prefix.
\tabularnewline
\hline
  Variable names 
&   
  should be short yet meaningful. The choice of a variable name should
  be mnemonic -- that is, designed to indicate to the casual observer
  the intent of its use.  One-character variable names should be
  avoided except for temporary ``throwaway'' variables. Common names
  for temporary variables are i, j, k, m, and n for integers; c, d,
  and e for characters. 
\tabularnewline
\hline
  Private and protected class members
&   
  should have a trailing underscore, like \T{my\_state\_}.
\tabularnewline
\hline
  \#defines
&
  should be all uppercase.
\tabularnewline
\hline
\end{tabular}

\subsection{Programming Practices}

\subsubsection{Adherence to Standards}

All code should adhere to the ISO C++ standards.
The presence or absence of library functions not specified by the
Posix standard on a particular platform should be detected by use of
Autoconf and appropriate logic emplaced to either work around the
absence or provide a good error message.

\subsubsection{Use typedef in preference to \#defines}

New types should be introduced via typedefs rather than by \#defines
-- these types are then visible in debuggers and the compiler can do
stronger type-checking.

\subsubsection{Use of const qualifier}

Where possible pointers should be passed using the const qualifier.
This is especially important for strings.

\subsubsection{Global and Static variables}

Don't make any variable global or static without good
reason.  Access to module level statics in other files can often
be granted via a function call rather than by making the variable global.

\subsubsection{Constants}

Numerical constants (literals) should not be coded directly, except
for -1, 0, and 1, or other numbers in loop counters.

\subsubsection{Variable Assignments}

Avoid assigning several variables to the same value in a single
statement. It is hard to read. Example:

\begin{quote}
\begin{verbatim}
fchar = lchar = 'c'; /* AVOID! */
\end{verbatim}
\end{quote}

Do not use the assignment operator in a place where it can be easily
confused with the equality operator. Example:

\begin{quote}
\begin{verbatim}
if (file = fopen(...)) 
{        /* AVOID! */
  ...
}

\end{verbatim}
\end{quote}

should be written as 

\begin{quote}
\begin{verbatim}
if ( NULL != (file = fopen (...)) ) 
{
  ...
}

\end{verbatim}

\end{quote}

Do not use embedded assignments in an attempt to improve run-time
performance. This is the job of the compiler. Example:

\begin{quote}
\begin{verbatim}

d = (a = b + c) + r;  /* AVOID! */

\end{verbatim}
\end{quote}

should be written as 

\begin{quote}
\begin{verbatim}

a = b + c;
d = a + r;

\end{verbatim}

\end{quote}

\subsubsection{Miscellaneous Practices}

\paragraph{Parentheses}

It is generally a good idea to use parentheses liberally in
expressions involving mixed operators to avoid operator precedence
problems. Even if the operator precedence seems clear to you, it might
not be to others-you shouldn't assume that other programmers know
precedence as well as you do.

\begin{quote}
\begin{verbatim}
if ( a == b && c == d )     /* AVOID! */
if ( (a == b) && (c == d) ) /* RIGHT */
\end{verbatim}
\end{quote}
 
\paragraph{Returning Values}

Try to make the structure of your program match the intent. Example:

\begin{quote}
\begin{verbatim}
if ( booleanExpression ) 
{
  return true;
} 
else 
{
  return false;
}
\end{verbatim}
\end{quote}

should instead be written as 

\begin{quote}
\begin{verbatim}

return booleanExpression;

\end{verbatim}
\end{quote}

\paragraph{Expressions before `?' in the Conditional Operator }

If an expression containing a binary operator appears before the ? in
the ternary ?: operator, it should be parenthesised. Example:

\begin{quote}
\begin{verbatim}
(x >= 0) ? x : -x;

\end{verbatim}
\end{quote}

\paragraph{Special Comments}

Use TODO in a comment to flag something that is bogus but works, or
missing pices. Use FIXME to flag something that is bogus and broken.

\paragraph{Compilation with warnings enabled}

It is recommended that developers compile with all warnings enabled.
Compiler warnings often flag dubious practices and common coding
errors.

\end{document}

