
 \subsection{Quick Introduction}

  The SAGA filesystem package provides an abstraction to remote
  file systems.  It inherits the SAGA namespace package, so the
  reader may want to study Section~\ref{sec:namespace} as well.

  File systems provide more than just a namespace: they also provide
  access to the \I{contents} of files:  the |saga::filesystem| package
  extends the name space package by adding |read()|, |write()| and
  |seek()| to the entries (i.e., to |saga::filesystem::file|):

  \begin{mycode}[label=Accessing a file]
  // allocate a 'large' buffer statically
  char mem[1024];
  saga::mutable_buffer buf (mem, 512);

  // open a file
  saga::url u ("/etc/passwd");
  saga::filesystem::file f (u);

  // read data into buffer - the first 512 bytes get fill
  f.read (buf);

  // seek file and buffer
  buf.set_data (mem + 512, 512);
  f.seek (123, saga::filesystem::Current);

  // read again
  f.read (buf);

  // print what we got
  std::cout << mem << std::endl;

  return (0);
  \end{mycode}

  The |saga::buffer| class represents the memory the data are written
  into.  The buffers' memory management can be controlled, but the above
  example leaves everything to the SAGA implementation---for
  details, see Section~\ref{sec:buffers}.

  Note that there are more I/O methods available in the filesystem
  package.  They are, however, mostly provided for optimization,  but
  are really \I{required} to make remote file I/O operations useful.

  Conceptually, the file remains a namespace entry with added read,
  write and seek capabilities.  For more information, see the details
  below, and also the section \I{''Namespaces''}
  (Sec.~\ref{sec:namespace}).
  
 \subsection{Reference}

  \begin{mycode}[label=saga::filesystem enums]
  namespace saga
  {
    namespace filesystem 
    {
      enum flags 
      {
        Unknown         = /*  -1, */  saga::name_space::Unknown       ,
        None            = /*   0, */  saga::name_space::None          ,
        Overwrite       = /*   1, */  saga::name_space::Overwrite     ,
        Recursive       = /*   2, */  saga::name_space::Recursive     ,
        Dereference     = /*   4, */  saga::name_space::Dereference   ,
        Create          = /*   8, */  saga::name_space::Create        ,
        Exclusive       = /*  16, */  saga::name_space::Exclusive     ,
        Lock            = /*  32, */  saga::name_space::Lock          ,
        CreateParents   = /*  64, */  saga::name_space::CreateParents ,
        Truncate        =    128, 
        Append          =    256,
        Read            = /* 512, */  saga::name_space::Read          ,
        Write           = /*1024, */  saga::name_space::Write         ,
        ReadWrite       = /*1536, */  saga::name_space::ReadWrite     ,
        Binary          =   2048 
      };       

      enum seek_mode
      {
        Start   = 1,
        Current = 2,
        End     = 3
      }; 
    }
  }
  \end{mycode}
    
  \begin{mycode}[label=Prototype: saga::filesystem::iovec]
  namespace saga
  {
    namespace filesystem 
    {
      class const_iovec 
          : public saga::const_buffer
      {
        public:
          const_iovec (void const  * data, 
                       saga::ssize_t size, 
                       saga::ssize_t len_in = -1);
         ~const_iovec (void);
  
          saga::ssize_t get_len_in  (void) const;
          saga::ssize_t get_len_out (void) const;
  
      };
  
      class iovec 
          : public saga::mutable_buffer
      {
        public:
          iovec (void *         data   = 0, 
                 saga::ssize_t  size   = -1, 
                 saga::ssize_t  len_in = -1, 
                 buffer_deleter cb     = default_buffer_deleter);
         ~iovec (void);
  
          void          set_len_in  (saga::ssize_t len_in);
          saga::ssize_t get_len_in  (void) const;
          saga::ssize_t get_len_out (void) const;
      };
    }
  }
  \end{mycode}

  \begin{mycode}[label=Prototype: saga::filesystem::file]
  namespace saga
  {
    namespace filesystem 
    {
      class file 
        : public saga::name_space::entry
      {
        public:
          file (saga::session const & s, 
                saga::url             url, 
                int                   mode = Read);
          file (saga::url url, 
                int                   mode = Read);
          file (saga::object const &  o);
          file (void);
          ~file (void);
  
          file & operator= (saga::object const & o);
  
          saga::off_t   get_size (void);
  
          saga::ssize_t read     (saga::mutable_buffer buffer, 
                                  saga::ssize_t        length = 0);
          saga::ssize_t write    (saga::const_buffer   buffer, 
                                  saga::ssize_t        length = 0);
          saga::off_t   seek     (saga::off_t          offset, 
                                  seek_mode            mode);
  
          void          read_v   (std::vector <iovec>  buffer_vec);
          void          write_v  (std::vector <const_iovec> 
                                  buffer_vec);
  
          saga::ssize_t size_p   (std::string          pattern);
          saga::ssize_t read_p   (std::string          pattern, 
                                  saga::mutable_buffer buffer);
          saga::ssize_t write_p  (std::string          pattern, 
                                  saga::const_buffer   buffer);
  
          std::vector <std::string> 
            modes_e  (void);
          saga::size_t  size_e   (std::string          ext_mode, 
                                  std::string          specification);
          saga::ssize_t read_e   (std::string          ext_mode, 
                                  std::string          specification,
                                  saga::mutable_buffer buffer);
          saga::ssize_t write_e  (std::string          ext_mode, 
                                  std::string          specification,
                                  saga::const_buffer   buffer);
      };
    }
  } 
  \end{mycode}

  \begin{mycode}[label=Prototype: saga::filesystem::directory]
  namespace saga 
  { 
    namespace filesystem 
    {
      class directory 
          : public saga::name_space::directory
      {
        public:
          directory (saga::session const & s, 
                     saga::url             url, 
                     int                   mode = ReadWrite);
          directory (saga::url             url, 
                     int                   mode = ReadWrite);
          directory (saga::object const &  o);
          directory (void);
         ~directory (void);
  
          directory & operator= (saga::object const & o);
  
          saga::off_t get_size (saga::url url);
          bool        is_file  (saga::url url);
          file        open     (saga::url url, 
                                int       flags = Read);
          directory   open_dir (saga::url url, 
                                int       flags = ReadWrite);
      }
    }
  }
  \end{mycode}

  %\subsection{Filesystem Details}
%
%   The described classes are syntactically and semantically
%   POSIX-oriented~\cite{posix1,posix2,posix3}.  Executing large
%   numbers of simple POSIX-like remote data access operations
%   is, however, prone to latency-related performance problems.
%   To allow for efficient implementations, the presented API
%   borrows ideas from GridFTP and other specifications which are
%   widely used for remote data access.  These extensions should
%   be seen as just that: optimizations.  Be aware that the SAGA
%   adaptors usually implement the POSIX-like |read()|, |write()|
%   and |seek()| methods, and rarely implement the additional
%   optimized methods (a 'NotImplemented' exception is thrown if
%   these are not implemented).  The optimizations included here
%   are:
%
%   \paragraph{Scattered I/O}
%   
%   Scattered I/O operations are already defined by POSIX, as
%   |readv()| and |writev()|.  Essentially, these methods
%   represent \B{v}ector versions of the standard POSIX
%   |read()/write()| methods; the arguments are,
%   basically,  vectors of instructions to
%   execute, and buffers to operate upon.  In other
%   words, |readv()| and |writev()| can be regarded as
%   specialized bulk methods, which cluster multiple I/O
%   operations into a single operation.  The advantages of such an
%   approach are that it is easy to implement, it is very close to
%   the original POSIX I/O in semantics, and, in some cases, it is even
%   very fast.  The disadvantage is that for many small I/O
%   operations (a common occurrence in SAGA use cases), the
%   description of the I/O operations can be larger than the
%   sent, returned or received data.
%
%   \paragraph{Pattern-Based I/O (FALLS)}
%   
%   One approach to address the bandwidth limitation of scattered
%   I/O is to describe the required I/O operations at a more
%   abstract level.  Regularly, repeating patterns of binary data
%   can be described by the so-called 'FAmiLy of Line Segments'
%   (FALLS)~\cite{falls}.  The pattern-based I/O routines in SAGA
%   use such descriptions to reduce the bandwidth limitation of
%   scattered I/O.  The advantage of such an approach is
%   that it targets very common data access patterns (at least
%   those commonly found in SAGA use cases).  The
%   disadvantages are that FALLS is a paradigm not widely known
%   or used, and that FALLS is by definition, limited to regular
%   patterns of data, hence the inefficiency for more
%   randomized data access.
%
%    \mywfig{r}{0.45}{falls}{\footnotesize The highlighted
%    elements are defined by \T{"(0,17,36,6,(0,0,2,6))"}.}
%
%   FALLS was originally introduced
%   for transformations in parallel computing. There is also a
%   parallel filesystem which uses FALLS to describe the file
%   layout. FALLS can be used to describe regular subsets of
%   arrays with a very compact syntax.  
%   
%   FALLS patterns are formed as 5-tuples:
%   |"(from,to,stride,rep,(pat))"|.  The |from| element defines
%   the starting offset for the first pattern unit; |to| defines
%   the finishing offset of the first pattern unit; |stride|
%   defines the distance between consecutive pattern units (start
%   to start); and |rep| defines the number of repetitions of the
%   pattern units.  The optional 5th element, |pat|, allows to
%   define nested patterns, where the internal pattern defines
%   the unit the outer pattern is applied to (by default it is
%   one byte).  As an example, the following FALLS describe the
%   highlighted elements of the matrix in Fig.~\ref{fig:falls}:
%   |"(0,17,36,6,(0,0,2,6))"|: the inner pattern describes a
%   pattern unit of one byte length (from 0 to 0), with a
%   distance of 2 to the next application, and 6 repetitions.
%   These are the 6 bytes per line which are marked.  The outer
%   pattern defines the repeated application of the inner
%   pattern, starting at 0, ending at 17 (end of line), distance
%   of 36 (to the beginning of next but one line), and repetition of 6.
%
%
%   \paragraph{Extended I/O}
%   
%   GridFTP (which was designed for a similar target domain)
%   introduced an additional remote I/O paradigm, that of
%   Extended I/O operations.
%   
%   In essence, the Extended I/O paradigm allows the formulation
%   of I/O requests using custom strings, which are not
%   interpreted on the client, but on the server side; these can
%   be expanded to arbitrarily complex sets of I/O operations.
%   The type of I/O request encoded in the string is called
%   |mode|.  A server may support one or many of these extended
%   I/O modes.  Whereas the approach is very flexible and
%   powerful and has proven its usability in GridFTP, a
%   disadvantage is that it requires very specific infrastructure
%   to function, i.e., it requires a remote server instance which
%   can interpret opaque client requests.  Additionally, no
%   client side checks or optimizations on the I/O requests are
%   possible.  Also, the application programmer needs to estimate
%   the size of the data to be returned in advance, which in some
%   cases is very difficult.\\
%   
%   The three described operations have, if compared to each
%   other, increasing semantic flexibility, and are increasingly
%   powerful for specific use cases.  However, they are also
%   increasingly difficult to implement and support in a generic
%   fashion.  It is up to the SAGA adaptors and the specific use
%   cases, to determine the level of I/O abstraction that serves
%   the application best and that can be best supported in the
%   target environment.


 \subsubsection*{Enum \T{flags}}

  The enum \T{flags} are inherited from the
  \T{namespace} package.  A number of file specific flags are
  added to it.  All added flags are used for the opening of
  \T{file} and \T{directory} instances, and are not applicable
  to the operations inherited from the \T{namespace} package.
  
    |Truncate|\\[0.3mm]
    \begin{tabular}{cp{110mm}}
      ~~ & Upon opening, the file is truncated to length 
           \T{0}, i.e., a following \T{read()} operation will 
           never find any data in the file.  That flag does not 
           apply to directories.
    \end{tabular}

    |Append|\\[0.3mm]
    \begin{tabular}{cp{110mm}}
      ~~ & Upon opening, the file pointer is set to the 
           end of the file, i.e., a following \T{write()} 
           operation will extend the size of the file.  That 
           flag does not apply to directories.
    \end{tabular}

    |Read|\\[0.3mm]
    \begin{tabular}{cp{110mm}}
      ~~ & The file or directory is opened for reading---that does not imply the ability to write to the 
           file or directory.
    \end{tabular}

    |Write|\\[0.3mm]
    \begin{tabular}{cp{110mm}}
      ~~ & The file or directory is opened for writing---that does not imply the ability to read from the 
           file or directory.
    \end{tabular}

    |ReadWrite|\\[0.3mm]
    \begin{tabular}{cp{110mm}}
      ~~ & The file or directory is opened for reading 
           and writing.
    \end{tabular}

    |Binary|\\[0.3mm]
    \begin{tabular}{cp{110mm}}
      ~~ & Some operating systems (notably windows-based systems) distinguish between binary and 
           non-binary modes---this flag mimics that behaviour.
    \end{tabular}


  \subsubsection*{Class \T{iovec}}

    The \T{iovec} class inherits the \T{saga::buffer}
    class, and three additional state attributes: \T{offset},
    \T{len\_in} and \T{len\_out} (with the latter one being
    read-only).  With that addition, the new class can be used
    very much the same way as the \T{iovec} structure defined by
    POSIX for \T{readv}/\T{writev}; the buffer \T{len\_in} is
    being interpreted as the POSIX \T{iov\_len}, i.e., the
    number of bytes to read/write.

    If \T{len\_in} is not specified, that length is set
    to the size of the buffer.  For
    application-managed buffers, it is a \T{BadParameter} error
    if \T{len\_in} is specified to be larger than size,  (see Section~\ref{ssec:buffer}
    for details on buffer memory management).  Before an
    \T{iovec} instance is used, its \T{len\_in} must be set to
    a non-zero value; otherwise its use will cause a
    \T{BadParameter} exception.

    After a \T{read\_v()} or \T{write\_v()} operation
    completes, \T{len\_out} will report the number of bytes
    read or written.  Before completion, the SAGA implementation will
    report \T{len\_out} to be \T{-1}.

