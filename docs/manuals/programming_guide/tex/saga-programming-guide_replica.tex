

 \subsection{Quick Introduction}

  Another package inheriting the namespace package is the replica
  management in |saga::replica|.  It introduces logical files,  which
  are namespace entries that have a number of locations (URLs)
  attached pointing to identical physical copies of the same file.
  The replica package is again a very simple extension of the namespace
  package: the class |logical_file| is a namespace |entry| class with
  a number of additional methods which allow to manage the list of
  attached URLs:

  \begin{mycode}[label=Managing replica locations]
  // open a logical file
  saga::url u ("/replica_1");
  saga::replica::logical_file lf (u, saga::replica::Create
                                  |  saga::replica::ReadWrite);

  // Add a replica location, replicate the file
  lf.add_location    (saga::url ("file://localhost//etc/passwd"));
  lf.replicate       (saga::url ("file://localhost//tmp/passwd"),
                      saga::replica::Overwrite);

  // list all locations 
  std::vector <saga::url> replicas = lf.list_locations ();

  for ( unsigned int i = 0; i < replicas.size (); i++ )
  {
    std::cout << "replica: " << replicas[i] << std::endl;
  }

  // remove the first location
  lf.remove_location (replicas[0]);
  \end{mycode}

  Managing replicas is fairly simple, because it only covers methods to
  list, add, delete, and update replica locations on logical files.
  Additionally, logical files and logical directories can have meta
  data attached, which are sets of string-based key-value
  pairs\footnote{For brevity, the example does not distinguish between scalar and
  vector attributes.}:

  \begin{mycode}[label=Managing replica meta data]
  // open a logical file
  saga::url u ("/replica_1");
  saga::replica::logical_file lf (u, saga::replica::Create
                                  |  saga::replica::ReadWrite);
  // get all attributes
  std::vector <std::string> keys = lf.list_attributes ();

  // print the keys and values
  for ( unsigned int i = 0; i < keys.size (); i++ )
  {
    std::string key = keys[i];
    std::string val;

    if ( lf.attribute_is_vector (key) )
    {
      std::vector <std::string> vals = lf.get_vector_attribute (key);
      val = vals[0] + " ...";
    }
    else
    {
      val = lf.get_attribute (key);
    }
    std::cout << key << " -> " << val << std::endl;
  }
  \end{mycode}

  The attentive reader will notice that the attribute management
  is in accordance with the attribute management part of the SAGA
  \LF.

 \subsection{Reference}

 \begin{mycode}[label=Prototypes: saga::replica]
  namespace saga
  {
    namespace replica
    {
      namespace metrics
      {
        char const * const logical_file_modified            
                   =      "logical_file.Modified";
        char const * const logical_file_deleted             
                   =      "logical_file.Deleted";
        char const * const logical_directory_created_entry  
                   =      "logical_directory.CreatedEntry";
        char const * const logical_directory_modified_entry 
                   =      "logical_directory.ModifiedEntry";
        char const * const logical_directory_deleted_entry  
                   =      "logical_directory.DeletedEntry";
      }

      enum flags
      {
        Unknown         = /*  -1, */  saga::name_space::Unknown       ,
        None            = /*   0, */  saga::name_space::None          ,
        Overwrite       = /*   1, */  saga::name_space::Overwrite     ,
        Recursive       = /*   2, */  saga::name_space::Recursive     ,
        Dereference     = /*   4, */  saga::name_space::Dereference   ,
        Create          = /*   8, */  saga::name_space::Create        ,
        Exclusive       = /*  16, */  saga::name_space::Exclusive     ,
        Lock            = /*  32, */  saga::name_space::Lock          ,
        CreateParents   = /*  64, */  saga::name_space::CreateParents ,
                          // 128,          reserved for Truncate
                          // 256,          reserved for Append
        Read            =    512,
        Write           =   1024,
        ReadWrite       =   1036,
                         // 2048,          reserved for Binary
      };

      class logical_file
          : public saga::name_space::entry,
            public saga::attributes
      {
        public:
            logical_file (session const      & s,
                          saga::url            url,
                          int                  mode = Read);
            logical_file (saga::url            url,
                          int                  mode = Read);
            logical_file (saga::object const & o);
            logical_file (void);
           ~logical_file (void);

            logical_file & operator= (saga::object const & o);

            void  add_location    (saga::url url);
            void  remove_location (saga::url url);
            void  update_location (saga::url old,
                                   saga::url new);
            std::vector<saga::url>
                  list_locations  (void);
            void  replicate       (saga::url url,
                                   int       flags = None);
      };

      class logical_directory
          : public saga::name_space::directory,
            public saga::attributes
      {
      public:
        logical_directory  (saga::session const & s,
                            saga::url             url,
                            int                   mode = ReadWrite);
        logical_directory  (saga::url             url,
                            int                   mode = ReadWrite);
        logical_directory  (saga::object const  & o);
        logical_directory  (void);
       ~logical_directory  (void);

        logical_directory & operator=(saga::object const& o);

        bool  is_file      (saga::url url);
        std::vector<saga::url>
              find         (std::string               name_pattern,
                            std::vector <std::string> key_pattern,
                            int                       flags = Recursive);

        saga::replica::logical_file
              open         (saga::url                 url,
                            int                       flags = Read);
        saga::replica::logical_directory
              open_dir     (saga::url                 url,
                            int                       flags = None);
      };
    }
  }
 \end{mycode}

 \subsection{Replica Details}

