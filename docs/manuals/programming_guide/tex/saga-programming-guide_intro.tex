
 What are Grids?  How can you use them?  Why would you need an API for
 doing so?  What makes SAGA different from other Grid APIs?

 These are some of the questions which we will try to answer in this
 introduction.  Don't expect a complete treatise on the subject though;
 in stead, we will refer the interested reader to the literature where
 appropriate.  Also, this introduction can safely be skipped by
 readers which are familiar with Grid programming, or even just with
 using Grids. 

 The text assumes that the reader has some familarity with the concept
 of distributed computing and programming of distributed
 applications.


 \subsection*{Grid Computing}

  In ``What is a Grid?'' \cite{checklist}, a three point checklist is
  provided.  Although by no means rigorous or complete, it provides a
  starting point. Following that, Grids are:

  \begin{shortenum}

   \item coordinated resources that are not subject to centralized
         control,

   \item use standard, open, general-purpose protocols and interfaces,

   \item deliver non-trivial qualities of service.

  \end{shortenum}



 \subsection*{Grid Standardization}

  A diverse Grid Computing community has emerged over the past few years; 
  the Open Grid Forum
  (OGF)~\cite{ogf} serves as a focus point for the Grid community
  groups and Grid standardization efforts.  The main topics of 
  standardization are many fold, and cover, amongst others, 

  \begin{shortlist}
   \item service architectures
   \item service interfaces
   \item wire protocols
   \item information models
   \item Grid related markup languages
   \item user interfaces
  \end{shortlist}

  For this document, as might be obvious, the last point, which includes
  Application Programming Interfaces (APIs), is of special interest.
  It should be noted that OGF's standardization mostly targets Grid
  middleware developers and Vendors, and that the OGF's API specifications
  are the top layer of OGF's standards stack.


 \subsection*{Grid APIs}

  Traditionally, most distributed (and Grid) middleware systems come
  with a 'native' API.  Best known for such an API is Globus, a
  pioneering Grid Middleware project: each version of Globus services was/is
  accompanied by an extensive API which allows programmers (end-user application 
  or otherwise), 
  to make use of these services, and provides access to Globus service client
  libraries, Globus protocol implementations, Globus security tools,
  Globus markup languages, etc.

  Despite the success of Globus, its approach exhibits a number of
  problems: applications written against a specific version of the
  Globus API are not (easily) portable to other versions of the API,
  and not portable at all to other Grid middleware, such as
  Genesis-II, GridSam, or Unicore.  Furthermore, typical middleware
  APIs expose a specific set of features available in that
  middleware. These do not necessarily match the Grid
  abstractions required by the various applications; higher level
  services which provide these abstractions, are then required
  to implement their own specific APIs.
  
  That situation is, to some extent, comparable to the 80's and
  90's, where many vendors of parallel computers shipped their
  specific proprietary communication library along with their
  product, to enable the end users to make use of the distributed
  compute power from within a single (distributed) application.  Before
  the establishment of MPI as a single dominant standard for this
  inter-node communication, the creation of \I{portable} distributed
  applications was an extremely difficult and tedious effort.  Only
  the adoption of MPI by basically all cluster vendors, and the
  \I{native support} for MPI on most platforms, could finally alleviate
  that problem\footnote{Note that MPI serves a specific problem space
  of tightly coupled massive parallel applications.  Loosely coupled
  parallel applications, for example, are served by a different set of
  distributed communication concepts, e.g., Peer2Peer systems, or
  component models, etc.}.


 \subsection*{SAGA}

  The Simple API for Grid Applications (SAGA), a proposed OGF
  standard, is in several ways a comparable effort to MPI: SAGA tries
  to establish a \I{single} API for Grid application programmers,
  which is ideally shipped with all Grid middleware, so that the
  programmer can focus on the application logic, instead of having to
  deal with tedious and complex middleware details.

  But SAGA tries to go further: instead of defining a fixed single
  common denominator, SAGA is by definition an extensible and modular
  effort. By being extensible both horizontally and vertically, SAGA
  should be able to adapt to a variety of use cases and user
  communities. Additionally, our SAGA implementation can be extended
  by additional middleware bindings, to support the widest possible
  portability for applications.


  \subsubsection*{Horizontal Extensibility of SAGA}

   The SAGA approach is extremely modular: a stable and finite core
   set of SAGA \LF packages is accompanied by a variable set of
   functional API packages.  That latter set is expected to grow over
   time, and to cover future emerging Grid programming paradigms.

   \X{graphics goes here}

   At the moment, the SAGA API covers the following functional
   packages:

   \begin{tabbing}
    XXXX      \= XXXXXXXXXX \= XXXXXXXXXXXXXXXXXXXXXXXXXXX \kill
    \> \T{jobs}:    \> job creation and management\\
    \> \T{files}:   \> interaction with file systems\\
    \> \T{replica}: \> management of logical files\\
    \> \T{streams}: \> BSD socket oriented IPC\\
    \> \T{rpc}:     \> remote procedure calls
   \end{tabbing}

   All these packages are provided by our SAGA implementation.
   Additional packages, which are in the process of being defined,
   include:
   
   \begin{tabbing}
    XXXX      \= XXXXXXXXXX \= XXXXXXXXXXXXXXXXXXXXXXXXXXX \kill
    \> \T{adverts}:  \> persistent storage of application level information\footnotemark[2]\\
    \> \T{sd}:       \> service discovery\footnotemark[2]\\
    \> \T{messages}: \> message based IPC\\
    \> \T{cpr}:      \> checkpoint and recovery\\
    \> \T{dais}:     \> database access and integration
   \end{tabbing}

   \footnotetext[2]{These packages are also provided in our
   implementation.}
  

  \subsubsection*{Vertical Extensibility of SAGA}

   The experiences in various OGF user communities show that the
   Grid programming models used in different application domains vary
   widely.  In particular, it seems impossible to completely
   standardize information models, data models, and data formats,
   without losing the abstractive power of a high level application
   oriented API.

   \X{graphics goes here}

   For that reason, SAGA tries to stay independent of these issues,
   and, at the same time, offers (a) the flexibility to natively
   accommodate a variety of data types and structures, and (b) the
   ability to additionally define higher level API packages, which
   address application domain specific services and programming
   models.  If and how these additional high level packages are 
   standardized is then up to the community creating these packages.
   SAGA, however, maintains a uniform and consistent API for a
   very wide variety of users and use cases, without being too specific
   and inflexible.

 
  \subsubsection*{Implementation Extensibility}

   Any Grid API is only as good and powerful as its underlying
   middleware is.  How does this statement hold for SAGA, which is by
   design not bound to a specific Grid middleware?  It holds true in
   fact: although the syntax and semantics of the SAGA API calls
   remain stable over the various Grid middlewares SAGA is running on
   top of, the specific performance characteristics, and in fact the
   exact set of SAGA calls provided, may vary from case to case.

   That may seem like a serious drawback, and probably is, but an API
   can do only so much to emulate missing middleware features.  In any
   case, to simplify the runtime portability of applications, our SAGA
   implementation allows binding to a variety of Grid middlewares
   \I{at runtime}, i.e., without the need to even relink your
   application.

   \X{graphics goes here}

   That functionality is provided by a plugin type mechanism, which
   loads middleware \I{adaptors} into SAGA as required.  These
   adaptors translate the SAGA API calls to the specific Grid
   middleware actions.  For more details on adaptors and their
   maintenance and configurations, please refer to the installation
   manual.


 \subsection{Getting Started}

  The reader should by now have a fair idea of the target scope of the
  SAGA API: it is designed to simplify the programming of novel
  applications which are to be run on a variety of Grid middlewares.

  The next chapter, the 'Quick Start Guide', should be enough to get
  you going, and to compile and run small SAGA applications.  The
  chapters after that will discuss the SAGA \LF in more detail, and
  also dive into the various functional packages we provide.  The chapters are independent, no specific reading order is implied, unless noted otherwise.

