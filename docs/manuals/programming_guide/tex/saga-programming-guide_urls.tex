
\subsection{Quick Introduction}

 URLs (and URIs, see below) are a dominant concept for referencing
 application-external resources.  As such, they are also widely used in
 the Grid world, and in SAGA.  The |saga::url| class helps to manage
 such URLs.

\subsection{Reference}

 \begin{mycode}[label=Prototype: saga::url]
  namespace saga 
  {
    class url 
        : public saga::object
    {
      public:
        url  (void);
        url  (std::string const & urlstr);

        ~url (void);

        url & operator= (std::string const & urlstr);

        std::string get_string   (void) const;

        std::string get_url      (void) const;
        void        set_url      (std::string const & url);

        std::string get_scheme   (void) const;
        void        set_scheme   (std::string const & scheme);

        std::string get_host     (void) const;
        void        set_host     (std::string const & host);

        int         get_port     (void) const;
        void        set_port     (int port);

        std::string get_fragment (void) const;
        void        set_fragment (std::string const & fragment);

        std::string get_path     (void) const;
        void        set_path     (std::string const & path);

        std::string get_userinfo (void) const;
        void        set_userinfo (std::string const & userinfo);

        saga::url translate    (std::string const & proto);
    };

  
    std::ostream& operator<< (std::ostream       & os, 
                              saga::url    const & u);
    std::istream& operator>> (std::istream       & is, 
                              saga::url          & u);
    bool          operator== (saga::url    const & lhs, 
                              saga::url    const & rhs);
    bool          operator<  (saga::url    const & lhs, 
                              saga::url    const & rhs);
  }
 \end{mycode}

\subsection{Details}

 In the time of the Internet, there are probably only few people
 in the first and second world who are not aware of the concept
 of URLs.  However, many miss the finer details of the 
 \I{Uniform Resource Locators}, and for a good reason: URLs are
 \I{designed} to hide a number of complexities from the user.  In
 order to make efficient use of the SAGA API, and of many Grid
 concepts in general, we will need a basic understanding of
 several key elements of URLs\footnote{Technically, a URL is a
 URI that, \I{``in addition to identifying a resource,
 [provides] a means of locating the resource by describing its
 primary access mechanism (e.g., its network ‘location’).''}
 (\I{''Cool URLs don't change''},
 \T{http://www.w3.org/Provider/Style/URI})  This document,
 however, uses the terms URL and URI interchangeably.}.

 \subsubsection{The Structure of URLs}


  RFC 3986\cite{rfc-3986} defines the generic syntax for URIs (and thus
  for URLs).  A URL consists of four parts:

  \sshift \T{<scheme> : <hierarchical part> [ ? <query> ] [ \# <fragment> ]}

  The \T{scheme} defines how the other parts of the URL are
  interpreted.  The remaining parts define what resource the URL
  points to, and possibly how to access that resource.

  The \T{hierarchical part} usually contains the following
  information: \T{userinfo}, \T{host}, \T{port}, and \T{path}.  This
  information specifies \B{where} the resource is to be
  found.

  The \T{saga::url} class defines setters and getters for most of
  these individual URL elements.  When setting one of these
  elements, SAGA will make sure that the result is a valid URL---otherwise the setter will decline the operation with a
  \T{BadParameter} exception:

  \begin{mycode}
  saga::url u  ("ftp://remote.host.net:1234/data/old.dat");

  u.set_host   ("local.host/net");
  u.set_path   ("/data/new.dat");
  \end{mycode}

  Line 3 would result in a \T{BadParameter} exception, as the
  host contains an invalid character.  Line 4 in the example above would
  change the URL to

   \shift \T{ftp://remote.host.net:1234/data/new.dat}




 \subsubsection{URLs in Grids}

  Now, in Grids we face the problem that different URLs may
  point to the same resource. For example, the following two URLs
  may refer to the same file:

   \shift \T{http://data.silo.net/data/joe\_doe/recent/abc.dat}\\
   \shift \T{ftp://data.silo.net/pub/joe\_doe/recent/abc.dat}

  How can the SAGA API handle these differences transparently?
  In short: it can't\footnote{For a detailed discussion see
  section 2.11 of the SAGA specification \cite{ogf-gfd-90}.}, however, it  can help.  For example, it allows to translate URLs
  from one scheme to the other.  The code snippet below \I{may}
  be able to repeat the printout from above\footnote{Please note that,
  at the moment, no adaptor implements \T{url::translate()}.}:

  % FIXME: the saga-a prototype seems wrong?
  \begin{mycode}
  // Translating URLs
  saga::url in("http://data.silo.net/data/joe_doe/recent/abc.dat");
  saga::url out = in.translate ("ftp://");

  std::cout <<  in.get_string () << std::endl;
  std::cout << out.get_string () << std::endl;
  \end{mycode}

  Also, SAGA allows you to use the special scheme \T{any://} as
  a placeholder.  If specified, the SAGA implementation tries to
  guess the correct scheme, and does the potentially required
  URL translation in the backround.

  \HINT{Please note that URL translation is a non-trivial and
  error prone process, so it is not a good idea to rely on it
  if you want to keep your application portable (which is a MUST
  in Grids!).  Wherever possible stick to the known schemes, and
  keep your transactions inside a single scheme and name space.}


