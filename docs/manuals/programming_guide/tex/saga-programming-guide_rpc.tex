
 \subsection{Quick Introduction}

  Remote Procedure Calls are the second IPC mechanism provided in
  SAGA.  The respective |saga::rpc| package is modeled after the
  GridRPC standard \cite{gridrpc}.  The concept could not be simpler:
  an rpc handle instance (|saga::rpc::rpc|) has a single method,
  |call()|, which invokes the respective remote operation.  The call
  accepts a stack of |In|, |Out|, and |InOut| parameters.

% \begin{mycode}[label=rpc example]
% #include <saga/saga.hpp>
% #include <vector>
%
% char* mat_mult (char* A, char* B)
% {
%   // create and initialize parameters
%   std::vector <saga::rpc::parameter> pars (2);
%
%   params[0].set_data    (A)
%   params[0].set_io_mode (saga::rpc::In)
%
%   params[1].set_data    (B)
%   params[1].set_io_mode (saga::rpc::In)
%
%   params[2].set_io_mode (saga::rpc::Out)
%
%   // initialize rpc handle
%   saga::rpc::rpc handle (saga::url ("gridrpc://matmult.net/rpc_call");
%
%   // call remote operation
%   handle.call (params);
%
%
%   // return data from output parameter
%   return (params[2].get_data ();
% }
% \end{mycode}



 \subsection{Reference}

 \begin{mycode}[label=Prototypes: saga::rpc]

  namespace saga
  {
    namespace rpc 
    {
      enum io_mode
      {
        Unknown = -1,
        In      =  1,
        Out     =  2,
        InOut   =  In | Out
      };
    
      class parameter 
          : public saga::mutable_buffer
      {
        public:
          parameter (void         * data = 0, 
                     saga::ssize_t  size = -1, 
                     io_mode        mode = In,
                     buffer_deleter cb   = default_buffer_deleter);
    
         ~parameter (void);
    
          io_mode get_mode() const;
      }; 
  
      class rpc 
          : public saga::object,
            public saga::permissions
      {
        public:
          rpc (saga::session const & s, 
               saga::url             name = saga::url ());
          rpc (std::string   const & name);
          rpc (void);
         ~rpc (void);
  
          void call  (std::vector <parameter> parameters);
          void close (double timeout = 0.0);
      }; 
    }
  }


\end{mycode}


\subsection{Details}

  The rpc class constructor is used to initialize the remote
  function handle.  Be aware that this process may involve
  connection setup, service discovery, and other remote
  interactions!  In the constructor, the remote procedure to be
  invoked is specified by a URL, with the syntax:

    \shift |gridrpc://server.net:1234/my_function|

  with the elements responding to:

  \begin{tabbing}
    \shift XXXXXXXXX     \= --~~ XXXXX  \= --~~ \kill
    \shift |gridrpc|     \> --~~ scheme \> --~~ identifying a grid rpc operation\\
    \shift |server.net|  \> --~~ server \> --~~ server host serving the rpc call\\
    \shift |1234|        \> --~~ port   \> --~~ contact point for the server\\
    \shift |my_function| \> --~~ name   \> --~~ name of the remote method to invoke
  \end{tabbing}
  
  All elements can be empty, which allows the implementation to
  fall back to invoke a default remote method.
  
  Furthermore, the rpc class offers one method, |call()|, which
  invokes the remote procedure, and returns the respective |Out|
  and |InOut| parameters.

  \begin{mycode}[label=Remote Procedure Call Example]
  {
    // initialize the rpc handle
    saga::rpc::rpc rm(saga::url ("rpc://remote.host.net:31415/get_pi"));

    // initialize the parameter stack
    saga::rpc::parameter pi (NULL, -1, saga::rpc::Out);
    std::vector <saga::rpc::parameter> parameters;
    parameters.push_back (pi);

    // invoke the remote procedure
    rm.call (parameters);

    // when completed, the output parameters are available
    std::cout << "Pi equals " << pi.get_data () << std::endl;
  }
  \end{mycode}
  
  The argument and return value handling is very basic, and
  reflects the traditional scheme for remote procedure calls,
  that is, an array of structures act as a parameter stack. For
  each element of the vector, the |parameter| struct describes
  its data |buffer|, the |size| of that buffer, and its
  input/output |mode|.
  
  The |mode| value has to be initialized for each |parameter|,
  and the |size| and |buffer| values have to be initialized for each
  |In| and |InOut| struct. For |Out| parameters, |size| may have
  the value |0|, in which case the |buffer| must be un-allocated. The |buffer|
  is to be created (e.g., allocated) by the SAGA
  implementation upon arrival of the result data, with a size
  that is sufficient to hold all result data.  The |size| value is then
  set by the implementation to be equal to the allocated buffer size.  

  \HINT{This argument handling scheme allows efficient
  (copy-free) passing of parameters, because the parameter vector are
  passed by reference.}

  When an |Out| or |InOut| struct uses a pre-allocated buffer,
  any returned data exceeding the buffer size are discarded.
  The application is responsible for specifying correct buffer
  sizes for pre-allocated buffers; otherwise, the behaviour is in
  general undefined.  For more details on buffer management,
  see Section~\ref{sec:buffers} about buffer management.

  

